%?????????????????????????
% Nombre: capitulo5.tex  
% 
% Texto del capitulo 5
%---------------------------------------------------

\chapter{Conclusi�n}
\label{conclusiones}
Tras la realizaci�n de la practica hemos podido constatar la potencia de los m�todos disponibles en R para el modelado y predicci�n de series temporales. La curva de aprendizaje no ser�a muy elevada, al menos, para un control b�sico de las herramientas y test disponibles. Por otro lado, constatar el potencial del estos modelos y las innumerables aplicaciones del mundo real y las cuales  un analista de datos deber�a conocer al menos en cierta medida.

Por otro lado, tras el an�lisis predictivo de los a�os anteriores en la serie diaria, es menester mencionar el car�cter de las series y su aparente dificultad de modelar cambios repentinos por ejemplo en el caso actual que nos compete de la temperatura, el modelo predice en funci�n de datos anteriores y en a�os anteriores encaja bastante bien, pero en el caso actual de una bajada brusca de las temperaturas comienza a fallar. Modelar estos casos, implicar�n un nivel muy avanzado en las series temporales.

Al margen de valoraciones personales, si nos centramos en los objetivos descritos al inicio de la memoria, podemos concluir que hemos cumplido con todos. Por un lado, con las explicaciones te�ricas dadas en el c�digo y durante la memoria podr�amos dar por estudiado y asentado todo el contenido referente a series temporales, y a�n mas sobre test estad�sticos y su aplicaci�n en ciencia de datos los cuales han quedado constatada su potencia al discernir entre que modelo se comportar�a mejor y posterior mente comprobarlo con test reales. El pre-procesado de datos tambi�n se ha llevado a cabo de manera optima y bien podr�a mejorarse a�n m�s realizando filtrados o alg�n tipo de selecci�n que evitar� varianzas elevadas. Finalmente, podemos concluir que los modelos generados son aceptables.

\pagebreak
\clearpage
%---------------------------------------------------