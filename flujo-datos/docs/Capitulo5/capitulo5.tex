%?????????????????????????
% Nombre: capitulo5.tex  
% 
% Texto del capitulo 5
%---------------------------------------------------

\chapter{Conclusi�n}
\label{conclusiones}


Tras la realizaci�n y comprensi�n de la pr�ctica cabe realizar una peque�a retrospectiva sobre los objetivos de la misma y las conclusiones obtenidas. Por un lado, destacar la potencia del software Moa y la cantidad de opciones de generaci�n de problemas de flujo de datos para su posterior an�lisis. Es un software que claramente tiene innumerables usos para la realizaci�n de experimentos en el �mbito de investigaci�n pero cuya extrapolaci�n a problemas reales quiz� fuera m�s complicada, al menos con lo asimilado en estas pr�cticas. Por otro lado, el software es poco intuitivo y hace que sea incluso mas amigable su utilizaci�n en linea de comandos que con la propia GUI. 

En cuanto a los objetivos de la pr�ctica creo que se han alcanzado todos, por un lado asentando los conceptos te�ricos de la asignatura y posteriormente con la puesta en marcha y posterior interpretaci�n de todos los experimentos propuestos.

\pagebreak
\clearpage
%---------------------------------------------------